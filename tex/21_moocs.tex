\section{Massive Open Online Courses}\label{section:moocs}

\moocs are large-scale online courses that combine education, entertainment, and social networking~\cite{pappano2012year}. They are typically free of charge and open to anyone who is interested. A \mooc is usually structured around a set of learning goals from a certain area of study, which are presented over a period of a few weeks. Topics are diverse and cover various field, including arts, \cs, medicine, philosophy, and social sciences. In order to synchronize the learning process, \moocs typically have a fixed beginning.

\moocs are usually not for credit. However, they often award digital badges or certificates that reward participants for their accomplishments~\cite{chauhan2014massive}. Despite these incentives, the percentage of enrolled \mooc participants who satisfy the criteria to earn a reward is typically low. The majority of courses have completion rates of less than ten percent~\cite{jordan2014initial}.

While the three most prominent \mooc providers Coursera\foo{https://www.coursera.org/}, edX\foo{https://www.edx.org/}, and Udacity\foo{https://www.udacity.com/} are all based in the United States, several European platforms, such as FutureLearn\foo{https://www.futurelearn.com/}, Iversity\foo{https://iversity.org/}, and OpenupEd\foo{http://www.openuped.eu/}, came into existence. By removing geographical location as a barrier, \moocs have the potential to make high-quality education accessible for everyone with decent Internet connectivity. In this way, \moocs can help people in the most remote regions of the planet to promote their careers and expand their intellectual and personal networks~\cite{pappano2012year}. Additionally, learning becomes more flexible since online courses are not bound to fixed locations and daytimes~\cite{neuhaus2014platform}. While idealists consider \moocs to be an educational revolution, able to provide free education to the disadvantaged around the world, research suggests that \moocs may primarily favor people who are already educationally privileged~\cite{jordan2014initial}.

A \mooc provides course materials plus assessment tools for independent studying~\cite{european2014support}. A \mooc's content is usually packaged into videos of a few minutes in length, following a highly efficient approach to convey information, since videos are dense in information but can be paused and reviewed at any point~\cite{fox2014software}. Video content is complemented by reading material as well as optional lessons for highly ambitious learners. Lectures are usually separated by self-tests that enable learners to reflect upon the knowledge they gained. Graded assignments are used as a means of performance evaluation.

In order to fit the massive educational context, \moocs' assignments must be evaluated in a scalable fashion. While modern web technologies enable scalable delivery of lecture videos as well as real-time learner collaboration and interaction, the abilities to evaluate complex student assignments remain limited~\cite{piech2013tuned}. As a consequence, \mooc assignments are usually restricted to quiz tasks that are easily automatically assessable, such as multiple-choice questions, fill-in-the-blank questions, and ordering tasks~\cite{willems2013introducing}.

Depending on the openness of their content and learning process, \moocs can be divided into two categories: x\moocs and c\moocs~\cite{grunewald2013designing}. The majority of \moocs belong to the group of x\moocs. They follow a traditional classroom model, including a predefined schedule, instructor-led video lectures, and graded exercises. Furthermore, x\moocs tend to use learning material with proprietary licenses. In contrast, c\moocs are based on open educational resources and rely on learners who actively create knowledge and shape the learning process through their interactions. c\moocs offer learners a more autonomous, self-organized, and social learning experience.

Most \mooc platforms offer collaboration tools, such as discussion forums, chats, and private groups, which enable course participants to discuss course contents with fellow learners and teaching assistants. These tools facilitate the creation of a virtual community of learners. Nevertheless, a common criticism of \moocs is that they are unable to replicate the social experience of traditional classroom education due to their massive size and sole reliance on technology~\cite{warren2014facilitating}.

These days, two years after the New York Times labeled 2012 the ``year of the \mooc''~\cite{pappano2012year}, \moocs are a widely recognized phenomenon. They continue to grow and are expected to see a rise in student registrations~\cite{chauhan2014massive}. Enrollment numbers up to six figures have attracted considerable media attention. However, such massive numbers are not representative for a typical \mooc. The average course has around 43,000 enrollments~\cite{jordan2014initial}. As the number of available courses is increasing, individual participant counts are expected to decline on average.
