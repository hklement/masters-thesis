\section{Hint Generation}\label{section:hint-generation1}

Learning programming is a trial-and-error process. As a part of this process, learners make mistakes, identify their misconceptions, and revise them. However, making mistakes can be discouraging if mistakes' origins remain unknown. In the event of a program error, standard interpreter output can be confusing for beginners and might not provide sufficient hints towards corrective actions. Common mistakes, such as confusing operators, using reserved keywords, or invoking methods with wrong arguments, might be easy to eliminate for advanced programmers, but they can be hard to identify for beginners~\cite{hristova2003identifying}. In the worst case, learners might lose their ambition to continue.

According to Truong et al.~\cite{truong2005learning}, it is essential that students are given the opportunity to practice in an environment where they can receive constructive and corrective feedback. Therefore, we want to provide learners with hints that facilitate the understanding of program errors. Such hints could explain an error in a more accessible way, supply context-specific information, or suggest specific corrective actions.

Hints have to be defined by instructors. In order to provide instructors with a guideline towards which errors are common, \tool logs and aggregates errors that occur during the execution of students' code. In order to associate context-specific error messages with certain error classes, errors are matched to hints using regular expressions, which allow ignoring insignificant textual differences, such as line numbers and identifier names. Hence, teachers have to provide a regular expression and a helpful message in order to define a hint.

In a large-scale e-learning environment, teachers cannot provide hints for every possible mistake, given the enormous number of code submissions and the large variety of errors. However, students who are solving the same assignments after having attended the same lectures make mistakes that tend to follow predictable patterns~\cite{singh2013automated}. Therefore, we think that addressing the most common errors allows covering a good share of mistakes with manageable effort.

Since instructors author hints, they are able to control frequency and explicitness. Novice learners might be provided with precise instructions on how to face a problem, whereas more advanced students might only be supplied with subtle clues. Such inexplicit hints point at the origin of a problem but enable learners to discover the solution themselves~\cite{vihavainen2012multi}. Alternatively, teachers are also free to decide against providing hints at all.
