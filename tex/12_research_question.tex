\section{Research Question}

This work deals with the question how \moocs can integrate practical programming assignments in a manner that meets the demands of novice programmers and satisfies the inherent scalability requirements of large-scale e-learning environments.

In order to answer this question, three important partial aspects are addressed. Firstly, we investigate how learners can be provided with a development environment for practicing programming that fits the needs of beginning programmers. Secondly, we explore how scalable execution and assessment of programming assignments can be achieved. Thirdly, we examine how the integration of practical programming exercises into existing e-learning systems, such as \glspl{lms} and \mooc platforms, can be facilitated. While providing a secure platform for the execution of learners' code is a relevant issue that is to be addressed in this thesis, in-depth security considerations are not within the scope of this work.

As a solution to the research question, \tool is presented. Being a web-based platform providing practical programming tasks for \moocs and other e-learning environments, \tool aims at facilitating the entry into programming and at attracting a diverse audience of interested learners. While the application is designed to be novice-friendly, it is not tailored to beginner-oriented programming paradigms. Rather, it is designed to support a wide range of various programming languages in a fashion that encourages novices yet is not limited to trivial programming tasks.
