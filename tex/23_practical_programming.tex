\section{Practice in Programming Education}\label{section:practical-programming}

\cs educators seem to agree that programming is best taught using a learning-by-doing approach. Learning to program does not only involve acquiring complex knowledge but also related practical skills~\cite{robins2003learning}. Therefore, gaining programming expertise requires rigorous practice~\cite{vihavainen2012multi}.

Programming assignments can help students to become familiar with programming languages and tools, and to understand how the principles of software design and development can be applied in practice~\cite{douce2005automatic}. On-campus programming courses usually make use of practical assignments that build up on theoretical content presented in lectures. These assignments are regarded to be an indispensable part of the educational framework~\cite{neuhaus2014platform} and are used for assessment by the majority of \cs academics~\cite{staubitz2014lightweight}. Feldman and Zelenski~\cite{feldman1996quest} believe that the major part of students' learning outcomes in a beginners' programming course originates from completing programming projects.

The most important deficits of novice programmers relate to designing problem solutions and express them as actual programs. Frequent practical programming exercises are a common way for addressing these issues~\cite{robins2003learning}. A complete solution to a programming task is considered to be an important step in building the confidence of student programmers~\cite{douce2005automatic}.

As introduced in Section~\ref{section:moocs}, the majority of \moocs are composed of static learning content, tools that facilitate social interaction, and quizzes, which are used for self-evaluation and assessment. As discussed earlier, the diversity of these quizzes is quite limited since grading has to be performed in a scalable fashion. Although multiple-choice questions can be helpful to reflect upon what the student has learned, this kind of assessment is no substitute for hands-on experience and practice when learning to program. According to Neuhaus et al.~\cite{neuhaus2014platform}, the current generation of \mooc platforms is well suited for presenting teaching material, but it provides only inadequate possibilities for hands-on experiments. Supported assignments are essentially non-interactive and do not allow a step-by-step development of solutions.

However, in order to enable a more holistic learning process, \moocs need to integrate activities that allow active experimentation and that relate to concrete experience~\cite{grunewald2013designing}. Willems et al.~\cite{willems2013introducing} state that the implementation of systems that allow the assessment of practical exercises can be a great challenge for course creators and platform designers. Nevertheless, the authors see the ability to offer classes with a high share of practical tasks and assignments as a key feature of \mooc platforms, which will have a crucial impact on a platform's competitive position.
