Programmierkenntnisse sind sehr gefragt; qualifiziertes Personal ist jedoch selten. Um dem Fachkräftemangel entgegenzuwirken, muss die Informatikausbildung ausgeweitet und breiter gefächert werden. \glslink{mooc}{Massive Open Online Courses (MOOCs)} weisen enormes Potenzial auf, einer großen und heterogenen Zielgruppe das Programmieren zu lehren. Jedoch bestehen sie zumeist aus Lehrvideos, Lesestoff und leicht zu bewertenden Quiz, was einer kompetenten Programmierausbildung nicht genügt. Um das Programmieren zu lehren, müssen \glslink{mooc}{MOOCs} dezidiert praktische Erfahrung vermitteln, Feedback zur Verfügung stellen und Bewertung von Programmen ermöglichen.

Diese Arbeit befasst sich mit der Fragestellung, wie \glslink{mooc}{MOOCs} praktische Programmieraufgaben in einer Weise integrieren können, die sowohl die Bedürfnisse von Programmieranfängern erfüllt als auch den Skalierbarkeitsanforderungen gerecht wird, welche großen E-Learning-Umgebungen innewohnen.

Möglichkeiten, \glslink{mooc}{MOOC}-Teilnehmer mit Mitteln zum praktischen Programmieren auszustatten, werden untersucht. Des Weiteren werden Anforderungen an Lösungen zur Programmierausbildung in großem Rahmen identifiziert.

Darauf aufbauend stellt diese Arbeit \tool vor, eine webbasierte Plattform für praktische Programmierübungen, welche für \glslink{mooc}{MOOCs}, die Programmieren lehren, konzipiert worden ist. Konzept und Implementierung der Plattform werden unter den Aspekten der Werkzeuge, welche Lernenden und Lehrenden geboten werden, isolierter und skalierbarer Code-Ausführung, skalierbarer Code-Bewertung sowie Interoperabilität diskutiert.
