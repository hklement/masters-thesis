Programming skills are highly demanded, but qualified employees are rare. In order to counteract the shortage of software engineers, \cs education has to be expanded and diversified. \Glspl{mooc} bear a tremendous potential for teaching programming to a large and diverse audience. However, they are usually composed of video lectures, reading material, and easily assessable quizzes, which is not sufficient for proper programming education. In order to teach programming, \moocs have to provide practice, feedback, and assessment of students' programs.

This work deals with the question how \moocs can integrate practical programming assignments in a manner that meets the demands of novice programmers and satisfies the inherent scalability requirements of large-scale e-learning environments.

Options for providing \mooc participants with the means for practicing programming are explored. In addition, requirements of a large-scale programming education solution are identified.

On this basis, this work introduces \tool, a web-based platform for practical programming exercises, which is designed to be used in \moocs that teach programming to beginners. The platform's concept and implementation are discussed with regard to tools provided to students and teachers, sandboxed and scalable code execution, scalable assessment, and interoperability.
