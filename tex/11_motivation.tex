\section{Motivation}\label{section:motivation}

In today's society, technological innovation plays an ever-increasing role for a country's development and economic growth~\cite{alhumoud2014using}. Technology touches virtually every part of our daily lives. As a consequence, programming abilities are required in many professional areas. Programming has become a key qualification of the \nth{21} century.

Although programming skills are highly demanded, qualified personnel are rare. The number of software engineering graduates is estimated to be far below the predicted need~\cite{vogel2014quality}, which is attributable to problems in both attracting and keeping students~\cite{blank2006robots}. Moreover, the participation of women and minorities in \gls{it} is dramatically low~\cite{blank2006robots,vogel2014quality}. Retention of first-year \cs students is a worldwide concern~\cite{lykke2014motivating} that has to be addressed.

Programming courses have high dropout rates and are generally considered to be difficult~\cite{lykke2014motivating,robins2003learning}. In order to succeed, students have to gain competencies in different areas, including syntax, semantics, algorithmic thinking, program design, and program comprehension~\cite{mueller2010learning}. Lack of early success is regarded to be one of the key issues leading to low success rates in introductory programming classes~\cite{truong2005learning}. In contrast, comfort level was found to have the strongest positive influence on success~\cite{wilson2001contributing}. Consequently, providing students with a comfortable and non-intimidating learning environment can help to decrease the number of university dropouts. In addition, interesting topics that offer space for creativity, such as robotics and mobile applications, can support students' motivation~\cite{petre2004using}.

In order to meet the increasing demand for \gls{it} professionals, raising completion rates is not enough, though. Governments and educational institutes have to promote \cs to a greater audience of young people~\cite{giordano2014use}. They are supported by non-profit organizations such as Code.org\foo{http://code.org/}, which aims at expanding \cs education in schools worldwide, inspiring more students, and improving diversity in \cs. Supported by well-known \gls{it} companies and celebrities, such as Bill Gates, Barack Obama, and Mark Zuckerberg, the initiative develops and distributes introductory programming courses and online video tutorials that are designed to be attractive to children.

In fact, children are more likely to develop an interest in technology if they are exposed to it at an early age~\cite{lau2009learning}. Introducing the young to programming in an appealing environment can help to inspire more children to engage in \cs~\cite{liyanagunawardena2014teaching}. Popular means for providing young people with motivating programming experience and for changing unfavorable misconceptions about \cs include summer camps~\cite{alhumoud2014using}, workshops~\cite{lau2009learning}, and after-school activities~\cite{maloney2008programming}, which often revolve around topics that are attractive to adolescents. In addition, there are specialized programming environments, aimed at teaching basic programming concepts in an entertaining way. Starter programming environments such as Alice\foo{http://www.alice.org/}, Greenfoot\foo{http://www.greenfoot.org/}, Scratch\foo{http://scratch.mit.edu/}, and ScratchJr\foo{http://www.scratchjr.org/}~\cite{fincher2010comparing} are usually visually attractive, interactive, and playful in order to appeal to children~\cite{giordano2014use}. Also, drag-and-drop programming approaches, which make first programming steps more accessible, are common.

A shortcoming of most local activities for promoting \cs is their limited capacity~\cite{loewis2014scaling}. More scalable \cs education can be provided by online programming courses. Web-based programming education platforms, such as Code School\foo{https://www.codeschool.com/}, Codecademy\foo{http://www.codecademy.com/}, and Khan Academy\foo{https://www.khanacademy.org/}, offer easy-to-start programming environments, which allow those who are keen to learn to get a taste of programming with a minimal initial threshold~\cite{vihavainen2012multi}. These platforms often make use of instructional videos, themed courses, and game mechanics, such as points and badges, to keep students motivated.

An equally scalable but more social learning experience is provided by \moocs, which make high-quality courses, covering various subjects, freely available to anyone connected to the Internet. Due to these properties, \moocs have a tremendous potential to introduce a large and diverse audience to the basics of programming. In fact, introductory courses in \cs and engineering, which are already offered by the majority of \mooc providers, are regarded to be an adequate means to attract students into the subject~\cite{vihavainen2012multi}. In addition, there are courses aimed at introducing teachers to new topics that can improve the appeal of their teachings~\cite{kay2014challenges}.

Google\foo{http://www.google.com/} recently published a guide for technical development\foo{http://www.google.com/edu/tools-and-solutions/guide-for-technical-development/} that suggests software engineering students to supplement their university knowledge and to develop technical skills through self-paced online learning by means of \moocs.

Within the scope of an initiative to create new jobs in the digital sector, the European Commission published a study~\cite{european2014support} investigating the demand and supply of \moocs related to web skills. Results of an associated survey show that \gls{it} professionals consider \moocs the best way to learn such abilities. The responses to the survey also indicate that learners are less interested in theoretical content but value practical experience. According to the study, neither the standard formulas of academic courses nor the prevalent \mooc format are optimal for teaching web-related skills. Instead, survey participants noted the importance of learning-by-doing practices.

Traditionally, \moocs are composed of video lectures, reading material, and assessment tools that are limited to a set of automatically gradable assignment types, such as quizzes. However, these means are not sufficient for teaching programming, which requires practice, feedback, and code assessment. In order to provide an attractive and supporting platform for teaching programming to the masses, \moocs have to fit the requirements of programming education. While \moocs can deliver course contents to tens of thousands of students, providing appropriate tools for practice and offering assessable practical programming assignments usually exceeds their built-in capabilities.
